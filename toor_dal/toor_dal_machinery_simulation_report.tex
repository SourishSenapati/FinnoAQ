\documentclass[11pt, a4paper]{article}
\usepackage[utf8]{inputenc}
\usepackage[T1]{fontenc}
\usepackage{geometry}
\usepackage{booktabs}
\usepackage{graphicx}
\usepackage{float}
\usepackage{color}
\usepackage{hyperref}
\usepackage{amsmath}
\usepackage{siunitx}
\usepackage{titlesec}
\usepackage{longtable}

% Page Geometry
\geometry{a4paper, margin=2.5cm}

% Branding Color
\definecolor{primary}{RGB}{0, 51, 102} % Dark Blue

% Title Setup
\title{\textbf{\LARGE TUR DAL MACHINERY SIMULATION AUDIT}\\
\large Rigorous Physics & Sensitivity Analysis Report\\
\small GPU-Accelerated Monte Carlo (N = 1,000,000)}
\author{\textbf{Sourish Senapati} \\ FINNO PROJECTS | AI Engineering Division}
\date{\today}

\begin{document}

\maketitle

\hrule
\vspace{1cm}

\begin{abstract}
This document serves as the technical validation appendix for the Toor Dal Manufacturing Project. It details the results of a 1,000,000-iteration Monte Carlo simulation executed on an NVIDIA RTX 4050 GPU. The simulation audited two competing architectures: the "Mixie Swarm" (Decentralized) and the "Ball Mill" (Centralized). \textbf{Conclusion:} While the Mixie Swarm is physically safe and optimal for R\&D/Lab scales ($<50$ kg/day), it fails the industrial reliability stress test due to downtime compounding. The \textbf{Ball Mill} is confirmed as the only viable architecture for mass production ($>500$ kg/hr).
\end{abstract}

\tableofcontents
\newpage

\section{Simulation Methodology}
The optimization engine utilized a rigorous physics-based approach:
\begin{itemize}
    \item \textbf{Thermal Physics:} Convection cooling dynamics ($h=15 W/m^2K$) and Arrhenius protein denaturation kinetics ($E_a=250 kJ/mol$).
    \item \textbf{Reliability Mathematics:} System downtime probability calculated as $P_{sys} = 1 - (1 - P_{unit})^N$, where $N=10$ for the Swarm.
    \item \textbf{Precision:} Double Precision (float64) was enforced to capture micro-probabilities of denaturation.
    \item \textbf{Scope:} 1,000,000 independent lifecycle scenarios.
\end{itemize}

\section{Phase 1: R\&D and Lab Scale Validation}
For small-batch production (1kg batches), the "Mixie Hack" was stress-tested.

\begin{table}[H]
    \centering
    \caption{R\&D Scale Simulation Results (1kg Batch)}
    \vspace{0.3cm}
    \begin{tabular}{l c}
        \toprule
        \textbf{Metric} & \textbf{Result} \\
        \midrule
        \textbf{Mean Protein Damage} & \textbf{0.0000\%} \\
        \textbf{95\% Worst-Case Temp Rise} & $+12.2^\circ$C \\
        \textbf{Optimal Protocol} & 3.6s ON / 1.0s OFF (Pulse Mode) \\
        \textbf{Mean Batch Cost} & INR 367.14 \\
        \bottomrule
    \end{tabular}
\end{table}

\textbf{Verdict:} The Mixie Swarm is \textbf{VALIDATED} for Lab/Pilot operations. The transient thermal physics confirms that the small mass dissipates heat fast enough to prevent denaturation, provided the strict pulse protocol is followed.

\section{Phase 2: Industrial Scale Stress Test}
For mass production (500 kg/hr), the model compared the Mixie Swarm against a standard Ball Mill under continuous operation stresses.

\subsection{Baseline Performance}
\begin{itemize}
    \item \textbf{Ball Mill Unit Cost:} INR 58.92 / kg (Stable)
    \item \textbf{Mixie Swarm Unit Cost:} INR 72.24 / kg (Volatile)
\end{itemize}

The Mixie Swarm suffers from a \textbf{Downtime Penalty}. With a unit failure rate of 2\%, the 10-unit cluster experiences a system degradation probability of $\approx 18.5\%$, driving up effective costs.

\subsection{Hard Stress Testing}
The architecture was subjected to four "Worst-Case" scenarios to determine robustness.

\begin{table}[H]
    \centering
    \caption{Hard Stress Test Results (Win/Loss Analysis)}
    \vspace{0.3cm}
    \begin{tabular}{l l l l}
        \toprule
        \textbf{Scenario} & \textbf{Ball Mill Cost} & \textbf{Mixie Cost} & \textbf{Winner} \\
        \midrule
        \textbf{A. High Failure (3\%)} & \textbf{INR 58.93 / kg} & INR 80.04 / kg & \textbf{BALL MILL} \\
        \textit{(Downtime Impact)} & \textit{0.40\%} & \textit{26.40\%} & \\
        \midrule
        \textbf{B. Elec. Shock (24 INR)} & \textbf{INR 60.72 / kg} & INR 74.03 / kg & \textbf{BALL MILL} \\
        \midrule
        \textbf{C. Heat Wave (40$^\circ$C)} & \textbf{INR 58.93 / kg} & INR 72.22 / kg & \textbf{BALL MILL} \\
        \midrule
        \textbf{D. Material Inflation} & \textbf{INR 69.10 / kg} & INR 84.68 / kg & \textbf{BALL MILL} \\
        \bottomrule
    \end{tabular}
\end{table}

\section{Sensitivity \& Dominance Analysis}
A Global Sensitivity Analysis (Sobol Approximation) identified the primary drivers of Unit Cost.

\begin{enumerate}
    \item \textbf{Raw Material Price (Corr: 0.69):} Dominant driver. Sourcing strategy is more critical than energy efficiency.
    \item \textbf{System Downtime (Corr: 0.19):} Moderate driver. This validates why the Swarm failed. Reliability is a significant cost factor.
    \item \textbf{Energy / Ambient / COP (Corr < 0.02):} Minor drivers. Optimization here yields diminishing returns.
\end{enumerate}

\section{Final Engineering Recommendations}

\subsection{Strategy: The "Dual-Phase" Approach}
\textbf{Phase 1 (Validation / Pilot):}
Deploy the \textbf{Mixie Swarm (10 Units)} for minimal CapEx (INR 1.2 Lakhs).
\begin{itemize}
    \item \textbf{Protocol:} 4s ON / 30s OFF.
    \item \textbf{Goal:} Market validation and recipe locking.
    \item \textbf{Risk:} High labor, frequent maintenance (acceptable at small scale).
\end{itemize}

\textbf{Phase 2 (Scale-Up):}
Upon reaching $>50$ kg/hr demand, immediately transition to a \textbf{Continuous Ball Mill}.
\begin{itemize}
    \item \textbf{Justification:} The Swarm becomes mathematically unviable due to compounding failure probabilities ($P_{sys} \to 100\%$).
    \item \textbf{OpEx Win:} The Ball Mill delivers a stable INR 58.92/kg cost, protecting margins against raw material volatility.
\end{itemize}

\vspace{2cm}
\footnotesize{Analysis generated via FINNO Intelligent Optimization Engine (Double Precision).}

\end{document}
