\documentclass[11pt, a4paper]{article}
\usepackage[utf8]{inputenc}
\usepackage[T1]{fontenc}
\usepackage{geometry}
\usepackage{booktabs}
\usepackage{graphicx}
\usepackage{float}
\usepackage{color}
\usepackage{hyperref}
\usepackage{amsmath}
\usepackage{siunitx}
\usepackage{titlesec}
\usepackage{longtable}

% Page Geometry
\geometry{a4paper, margin=2.5cm}

% Branding Color
\definecolor{primary}{RGB}{0, 51, 102} % Dark Blue

% Title Setup
\title{\textbf{\LARGE MASTER INDUSTRIAL OPTIMIZATION REPORT}\\
\large Tur Dal (Cajanus cajan) Analogue Manufacturing Line\\
\small GPU-Accelerated Monte Carlo Simulation (1,000,000 Iterations)}
\author{\textbf{Sourish Senapati} \\ FINNO PROJECTS | AI Optimization Division}
\date{\today}

\begin{document}

\maketitle

\hrule
\vspace{1cm}

\begin{abstract}
This report details the final optimization of the Tur Dal manufacturing process, executed on an NVIDIA RTX 4050 GPU using 1 million Monte Carlo simulations. The objective function defined was to maximize \textbf{Kilograms of Output per Rupee of Total Lifecycle Cost}. The simulation audited two distinct architectures: the "Mixie Swarm" (Decentralized) and the "Ball Mill" (Centralized). \textbf{Verdict:} The Mixie Swarm is optimal for R\&D and Pilot scales ($<50$ kg/hr) due to low CapEx. However, for industrial mass production ($>500$ kg/hr), the \textbf{Ball Mill} is the only viable solution, delivering a unit cost of \textbf{INR 58.92/kg} with 99.6\% reliability, whereas the Swarm failed stress tests with $>$18\% downtime.
\end{abstract}

\tableofcontents
\newpage

\section{Executive Summary}
The simulation converged on a **Dual-Phase Strategy** that optimizes capital efficiency during startup while ensuring operational stability at scale.

\subsection{Key Findings}
\begin{itemize}
    \item \textbf{Optimal Industrial configuration:} Continuous Ball Mill + Heat Pump Dryer.
    \item \textbf{Industrial Unit Cost:} INR 58.92 / kg (Lowest manageable cost).
    \item \textbf{R\&D Strategy:} "Mixie Swarm" validated for Lab/Pilot scale (0-50 kg/hr).
    \item \textbf{Risk Alert:} The "Mixie Hack" at industrial scale introduces unacceptable downtime risks (18-26\%), negating its CapEx advantage.
\end{itemize}

\section{Methodology}
\subsection{Simulation Engine}
\begin{itemize}
    \item \textbf{Hardware:} NVIDIA GeForce RTX 4050 Laptop GPU
    \item \textbf{Algorithm:} Vectorized PyTorch Monte Carlo Simulation
    \item \textbf{Sample Size:} $N = 1,000,000$ independent lifecycle scenarios
    \item \textbf{Constraints:} Protein Denaturation $< 0.1\%$, Moisture Variance $< 1.0\%$
\end{itemize}

\section{Detailed Machinery Optimization}

\subsection{1. Grinding Stage: The "Scale-Dependent" Strategy}
The simulation compared three distinct architectures. Stress testing revealed that optimal equipment choice depends entirely on production scale.

\begin{table}[H]
    \centering
    \caption{Grinding Architecture Comparison (Simulated Risk Adjusted)}
    \vspace{0.3cm}
    \begin{tabular}{l c c c}
        \toprule
        \textbf{Parameter} & \textbf{Ball Mill (Ind.)} & \textbf{Hammer Mill} & \textbf{Mixie Swarm (Lab)} \\
        \midrule
        \textbf{CapEx (INR)} & 4,50,000 & 1,50,000 & \textbf{1,20,000 (10x)} \\
        \textbf{Mean Unit Cost} & \textbf{INR 58.92 / kg} & INR 79.36 / kg & INR 72.24 / kg \\
        \textbf{System Downtime} & \textbf{0.40\%} & 1.30\% & \textbf{18.45\%} \\
        \textbf{Protein Damage} & 0.0\% & 12.0\% & 0.0\% \\
        \bottomrule
    \end{tabular}
\end{table}

\textbf{Failure Analysis:}
\begin{itemize}
    \item \textbf{Ball Mill:} Extremely robust. High CapEx is amortized effectively at 500kg/hr.
    \item \textbf{Mixie Swarm:} While physically capable of producing quality flour (0\% damage), the probabilistic downtime of 10 independent units causes chaotic production stops at industrial scale.
    \item \textbf{Recommendation:} Use Mixies ONLY for Pilot/R\&D. Use Ball Mill for Factory.
\end{itemize}

\subsection{2. Drying Stage: Energy Efficiency}
Drying is the largest OpEx driver. The simulation blindly favored Heat Pump technology despite higher CapEx.

\begin{itemize}
    \item \textbf{Electric Resistive Heater:} COP 0.9. Energy Cost: INR 18.00/kg (Prohibitive).
    \item \textbf{Heat Pump Dryer:} COP 3.5. Energy Cost: INR 4.60/kg.
    \item \textbf{Recommendation:} Invest the savings from R\&D phase into a high-quality Heat Pump Dryer.
\end{itemize}

\section{Risk Profile \& mitigation}

\subsection{Raw Material Volatility (Dominant Risk)}
Sensitivity analysis confirms that Raw Material Price (Correlation 0.69) is the #1 driver of unit cost.
\begin{itemize}
    \item \textbf{Impact:} A 10\% rise in Tur Khanda price increases Unit Cost by 8\%.
    \item \textbf{Mitigation:} Long-term supplier contracts are more valuable than machinery hacks.
\end{itemize}

\subsection{Mechanical Reliability}
\textbf{Risk:} Operations grinding to a halt. \\
\textbf{Mitigation:} The transition to Ball Mill at Scale $> 50$kg/hr reduces downtime risk from 18\% to 0.4\%.

\section{Final Configuration: The "Smart-Line" (Industrial Phase)}

\begin{longtable}{l l r}
\caption{Optimized Bill of Materials (500 kg/hr Plant)} \\
\toprule
\textbf{Stage} & \textbf{Equipment / Setting} & \textbf{Cost / Spec} \\
\midrule
\endfirsthead
\midrule
\textbf{Stage} & \textbf{Equipment / Setting} & \textbf{Cost / Spec} \\
\midrule
\endhead
1. Cleaning & Mechanical Grader + Destoner & INR 45,000 \\
2. Grinding & \textbf{Continuous Ball Mill (7.5 kW)} & INR 4,50,000 \\
   \textit{Setting} & \textit{RPM: Optimized Low Speed} & \textit{Cooling: Passive} \\
3. Mixing & Ribbon Blender (Horizontal) & INR 65,000 \\
4. Extrusion & Cold Extruder (Single Screw) & INR 3,50,000 \\
5. Drying & \textbf{Heat Pump Dryer (500kg)} & INR 4,50,000 \\
6. Packaging & FFS Pneumatic Packer & INR 1,20,000 \\
\midrule
\textbf{metrics} & & \\
\textbf{Throughput} & \textbf{499.6 kg/hr} & \\ 
\textbf{Total CapEx} & \textbf{INR 14.8 Lakhs} & \\
\textbf{Unit Cost} & \textbf{INR 58.92 / kg} & \\
\textbf{Net Reliability} & \textbf{99.6\%} & \\
\bottomrule
\end{longtable}

\section{Financial Projections (Annual)}
Assuming 3000 operational hours (1 shift):

\begin{itemize}
    \item \textbf{Total Output:} 1,499 Tonnes
    \item \textbf{Gross Revenue (INR 90/kg wholesale):} \textbf{INR 13.49 Crores}
    \item \textbf{Total Cost (INR 58.92/kg):} INR 8.83 Crores
    \item \textbf{Net Profit:} \textbf{INR 4.66 Crores}
    \item \textbf{ROI:} $>$ 250\% in Year 1
\end{itemize}

\section{Innovation: The R\&D Pivot}
The simulations saved the project from a potentially fatal mistake. Initially, the "Mixie Swarm" looked cheaper. Deeper analysis revealed it would cost \textbf{INR 13-20 Lakhs annually} in lost production time. The simulation successfully pivoted the strategy to a \textbf{Dual-Phase approach}, securing the low-cost entry of the Mixie for R\&D, but insisting on the Ball Mill for profitable scaling.

\vspace{2cm}
\footnotesize{Report generated via FINNO Intelligent Optimization Engine (Double Precision).}

\end{document}
